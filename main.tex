\documentclass[sigconf]{acmart}
\usepackage{graphicx}
\usepackage{hyperref}
\usepackage{amsmath, amssymb}
\usepackage{fancyhdr}
\usepackage{lipsum} % For dummy text

\title{Apartment Visitor Management System}
\author{Rohit Raghuwanshi}
\affiliation{
  \institution{IIT Bhilai}
  \city{Durg}
  \country{India}
}
\email{rohitrg@iitbhilai.ac.in}

\begin{document}

\maketitle

% Abstract Section
\section*{Abstract}
The Apartment Visitor Management System is a digital solution designed to enhance security and convenience in residential complexes. This system allows residents to pre-approve guests or delivery personnel, enabling seamless entry while ensuring security guards can log and verify visitor details. With real-time digital entry logs and an admin dashboard for apartment managers, the system improves security monitoring and minimizes unauthorized access.

\section{Introduction}
Visitor management in apartment complexes is a crucial aspect of security and convenience. Traditional methods often rely on manual logbooks, which pose security risks and inefficiencies. A modern digital system can address these challenges and streamline visitor approvals and tracking.

\begin{figure}[h]
\centering
\includegraphics[width=0.4\textwidth]{image1.png}
\caption{Example of a Digital Visitor Management System}
\label{fig:visitor_system}
\end{figure}

\section{Problem Motivation}
Traditional visitor management in apartment complexes relies on manual logbooks, which are prone to errors, security breaches, and inefficiencies. Manual logs lack real-time tracking, making it difficult to monitor visitor patterns and ensure security compliance. Furthermore, residents often face inconveniences in approving guests, leading to delays and miscommunication. A digital system can streamline the process, providing better security, accountability, and user experience.

\section{Challenges}
\begin{itemize}
    \item Ensuring real-time data synchronization between security personnel, residents, and administrators.
    \item Developing a user-friendly interface for both security guards (tablet-based entry logging) and residents (mobile/web app).
    \item Implementing a secure authentication system to prevent unauthorized access and data breaches.
    \item Managing large-scale data efficiently for different apartment complexes.
    \item Ensuring the system works smoothly across multiple devices and network conditions.
\end{itemize}

\section{Past Solution Approaches and Gaps}
\begin{itemize}
    \item \textbf{Manual Logbooks:} Susceptible to human errors and unauthorized modifications, difficult to retrieve historical data, and lack real-time monitoring.
    \item \textbf{Basic Digital Registers (Spreadsheets, Google Forms):} No automated visitor verification, requires manual entries by security personnel, lacks integration with mobile authentication for residents.
    \item \textbf{Standalone Mobile Apps:} Focus only on logging visitor details but lack seamless integration between security, residents, and administrators. Data storage is often limited and does not provide insights into visitor trends.
\end{itemize}

\begin{figure}[h]
\centering
\includegraphics[width=0.2\textwidth]{image.png}
\caption{Traditional Visitor Logbook System}
\label{fig:logbook}
\end{figure}

\section{Proposed Solution Approach}
The proposed system is a \textbf{web-based and mobile-responsive} visitor management platform that connects security personnel, residents, and apartment managers. The key components include:
\begin{itemize}
    \item \textbf{Resident Portal:} Residents can pre-approve visitors and receive real-time notifications when guests arrive.
    \item \textbf{Security Panel:} Guards can check visitor approvals, log new entries, and validate visitor identity using a tablet/web interface.
    \item \textbf{Admin Dashboard:} Apartment managers can monitor visitor trends, generate reports, and manage user access.
    \item \textbf{Database \& Cloud Integration:} A secure backend using \textbf{Node.js/Flask with MySQL/PostgreSQL} ensures efficient data management and scalability.
    \item \textbf{Authentication System:} Secure login with \textbf{OTP-based verification or biometric authentication} to prevent unauthorized access.
\end{itemize}

\section{Data Collection Strategy}
\begin{itemize}
    \item \textbf{Visitor Data:} Name, phone number, purpose of visit, apartment number, time of entry/exit.
    \item \textbf{Resident Data:} Name, apartment details, visitor approval status.
    \item \textbf{Security Logs:} Timestamped entry logs, guard approvals, suspicious activity reports.
    \item \textbf{Analytics Data:} Number of visitors per day, peak hours, repeated visitor patterns.
\end{itemize}

\section{Data Cleaning, Pre-processing, and Modeling Strategy}
\begin{itemize}
    \item \textbf{Data Validation:} Ensuring phone numbers are correctly formatted, verifying resident approval before granting entry.
    \item \textbf{Data Cleaning:} Removing duplicate visitor entries, correcting incorrect timestamps using a time-based validation process.
    \item \textbf{Pre-processing:} Standardizing names and apartment numbers for uniformity, implementing a hashing mechanism for secure storage of visitor credentials.
    \item \textbf{Modeling Strategy:} Using database indexing to optimize visitor search queries, implementing predictive analytics to identify visitor trends (e.g., common visiting hours), creating role-based access control to ensure proper authorization at each level (resident, guard, admin).
\end{itemize}

\section{Conclusion}
The Apartment Visitor Management System aims to modernize and secure the visitor tracking process within apartment complexes. By implementing a digital platform, this solution enhances security, minimizes unauthorized access, and improves overall efficiency. Future work involves expanding the system’s capabilities with AI-powered visitor anomaly detection and seamless integration with smart home security systems.

\section{References}
\begin{thebibliography}{9}
\bibitem{visitor_security} J. Smith, ``Security Enhancements in Residential Visitor Management,'' in \textit{Journal of Smart Cities}, vol. 12, no. 3, pp. 45-58, 2021.
\bibitem{cloud_security} R. Kumar, ``Cloud-based Security Solutions for Visitor Management,'' in \textit{International Conference on Cybersecurity}, 2020.
\bibitem{iot_management} L. Zhang, ``IoT-based Visitor Management in Gated Communities,'' in \textit{Proceedings of the IEEE Conference on Smart Homes}, 2019.
\end{thebibliography}

\end{document}