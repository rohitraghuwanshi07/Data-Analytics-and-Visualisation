\documentclass{article}
\usepackage{amsmath}
\usepackage{amsfonts}
\usepackage{titlesec}

% Set font size for section titles
\titleformat{\section}{\huge\bfseries}

\begin{document}

\begin{center}
    
    {\Huge ROHIT RAGHUWANSHI(12341820)}\\

    {\Large QUESTION NO 5.5}\\
\end{center}

% Functions
\section*{Functions}
1. \( f_1(x) = \sin(x_1) \cos(x_2), \quad x \in \mathbb{R}^2 \)

2. \( f_2(x, y) = x^{\top} y, \quad x, y \in \mathbb{R}^n \)

3. \( f_3(x) = xx^{\top}, \quad x \in \mathbb{R}^n \)

% Part a: Dimensions of Partial Derivatives
\subsection*{Part a: Dimensions of } \( \frac{\partial f_i}{\partial x} \)
1. For \( f_1(x) \): 
   - \( x \in \mathbb{R}^2 \) implies \( x_1 \) and \( x_2 \) are scalars.
   - Dimension of \( \frac{\partial f_1}{\partial x} \): \( \mathbb{R}^2 \) (1 output, 2 inputs).

2. For \( f_2(x, y) \): 
   - \( x, y \in \mathbb{R}^n \) implies \( x^{\top} y \) is a scalar.
   - Dimension of \( \frac{\partial f_2}{\partial x} \): \( \mathbb{R}^{n} \) (1 output, n inputs).

3. For \( f_3(x) \): 
   - \( xx^{\top} \) results in an \( n \times n \) matrix.
   - Dimension of \( \frac{\partial f_3}{\partial x} \): \( \mathbb{R}^{n \times n} \) (n outputs, n inputs).

% Part b: Jacobians
\subsection*{Part b: Compute the Jacobians}
1. For \( f_1(x) \):
   \[
   J_{f_1} = \begin{bmatrix}
   \frac{\partial f_1}{\partial x_1} & \frac{\partial f_1}{\partial x_2}
   \end{bmatrix} = \begin{bmatrix}
   \cos(x_1) \cos(x_2) & -\sin(x_1) \sin(x_2)
   \end{bmatrix}
   \]
   Dimension: \( 1 \times 2 \).

2. For \( f_2(x, y) \):
   \[
   J_{f_2} = \begin{bmatrix}
   \frac{\partial f_2}{\partial x_1} & \cdots & \frac{\partial f_2}{\partial x_n} \\
   \frac{\partial f_2}{\partial y_1} & \cdots & \frac{\partial f_2}{\partial y_n}
   \end{bmatrix} = 
   \begin{bmatrix}
   y^{\top} & x^{\top} \\
   x^{\top} & 0
   \end{bmatrix}
   \]
   Dimension: \( 1 \times n \).

3. For \( f_3(x) \):
   \[
   J_{f_3} = \frac{\partial (xx^{\top})}{\partial x} = 2x\mathrm{d}x^{\top} \quad \text{(using the product rule)}
   \]
   Dimension: \( n \times n \).

% Large Heading for Question No 5.8
\begin{center}
    {\Large QUESTION NO 5.8}\\
   \end{center}

% Part a
\section*{Part a}
Compute the derivative \( \frac{df}{dx} \) using the chain rule. 

The function is given as:
\[
f(z) = \exp\left(-\frac{1}{2} z\right), \quad z = g(y) = S^{-1} y, \quad y = h(x) = x - \mu
\]

\subsection*{Step 1: Understanding the Components}
\begin{itemize}
    \item \( x, \mu \in \mathbb{R}^D \)
    \item \( S \in \mathbb{R}^{D \times D} \)
    \item \( y = h(x) \): dimension \( D \)
    \item \( z = g(y) \): dimension \( D \)
    \item \( f(z) \): dimension \( \mathbb{R} \)
\end{itemize}

\subsection*{Step 2: Applying the Chain Rule}
\[
\frac{df}{dx} = \frac{df}{dz} \cdot \frac{dz}{dy} \cdot \frac{dy}{dx}
\]

\subsection*{Step 3: Compute Each Partial Derivative}
1. \( \frac{df}{dz} = -\frac{1}{2} \exp\left(-\frac{1}{2} z\right) \)
   Dimension: \( \mathbb{R} \)
   
2. \( \frac{dz}{dy} = S^{-1} \)
   Dimension: \( \mathbb{R}^{D \times D} \)

3. \( \frac{dy}{dx} = I \)
   Dimension: \( \mathbb{R}^{D \times D} \)

\subsection*{Step 4: Final Derivative}
\[
\frac{df}{dx} = \frac{1}{2} f(z) S^{-1}
\]

% Part b
\section*{Part b}
Compute the derivative \( \frac{df}{dx} \) for
\[
f(x) = \text{tr}(xx^{\top} + \sigma^2 I), \quad x \in \mathbb{R}^D
\]

\subsection*{Step 1: Understanding the Components}
\begin{itemize}
    \item \( xx^{\top} \): matrix dimension \( D \times D \)
    \item \( I \): identity matrix, dimension \( D \times D \)
    \item \( f(x) \) has dimension \( \mathbb{R} \).
\end{itemize}

\subsection*{Step 2: Applying the Derivative}
\[
\frac{df}{dx} = \frac{d}{dx} \text{tr}(xx^{\top}) = 2x^{\top}
\]
Dimension: \( \mathbb{R}^D \)

% Part c
\section*{Part c}
Compute the derivative \( \frac{df}{dx} \) using the chain rule for
\[
f = \tanh(z), \quad z = Ax + b, \quad x \in \mathbb{R}^N, \quad A \in \mathbb{R}^{M \times N}, \quad b \in \mathbb{R}^M
\]

\subsection*{Step 1: Understanding the Components}
\begin{itemize}
    \item \( x \): dimension \( N \)
    \item \( z \): dimension \( M \)
    \item \( f \): dimension \( M \)
\end{itemize}

\subsection*{Step 2: Applying the Chain Rule}
\[
\frac{df}{dx} = \frac{df}{dz} \cdot \frac{dz}{dx}
\]

\subsection*{Step 3: Compute Each Partial Derivative}
1. \( \frac{df}{dz} = \text{diag}(1 - \tanh^2(z)) \)
   Dimension: \( M \times M \)

2. \( \frac{dz}{dx} = A \)
   Dimension: \( M \times N \)

\subsection*{Step 4: Final Derivative}
\[
\frac{df}{dx} = \text{diag}(1 - \tanh^2(z)) A
\]
Dimension: \( M \times N \)

\end{document}